\documentclass{article}
\usepackage[left=2cm, right=2cm, top=0cm]{geometry}
\usepackage{amsmath}
\usepackage{amssymb}
\usepackage{fancyvrb}
\usepackage{xcolor}

\usepackage{graphicx}
\usepackage{enumitem}
\usepackage{tikz}
\usepackage{pgfplots}
\usepackage{hyperref}
\setlength\parindent{0pt}
% \hypersetup{
%    colorlinks,
%    citecolor=green,
%    filecolor=black,
%    linkcolor=blue,
%    urlcolor=blue
%}
\begin{document}
\title{LBA}
\author{Jacob Puthipiroj, Uyen Nguyen}

%\date{}
\maketitle

\abstract{In this paper, we establish a link between household gun ownership and teen suicide rates. Methodologically we establish this link through a simple linear regression of household gun ownership at the state level of suicides, and then a multiple regression, with the inclusion of suicide attempts as an additional explanatory variable. According to our model,  we predict that each 1\% absolute increase in household gun ownership is associated with a 2.15\% relative increase in youth suicides. Because guns are a much more lethal form of suicide attempt, a policy restricting the overall availability of guns should decrease not the number of suicide attempts, but the number of suicides overall. We therefore immediately urge policymakers to restrict the possession of firearms.}



Despite the limitations in our study, we find the usefulness of household gun ownership as a predictor of youth suicides at the state level. Specifically, we predict that each 1\% absolute increase in household gun ownership is associated with a 2.15\% relative increase in youth suicides. We hypothesize that the increased prevalence of guns presents a highly lethal option for suicidal teens in order to commit suicide, and thus would suggest public policies to more heavily regulate and reduce access to guns for teens (such as stringent universal background checks, junk-gun bans, permitless carry laws, or even buyback programs) in order to reduce teen mortality.\\


Word Count: 1334, 7 paragraphs, 3 visuals.


\end{document}